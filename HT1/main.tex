\input{Configuraciones/paquetes}

%--------------------------

\begin{document}
\input{Configuraciones/nombres}
%--------------------------

\begin{problema}
	Suponga que $f \geq 0, f$ es continua en $[a, b], \mathrm{y} \int_{a}^{b} f(x) d x=0 .$ Demuestre que $f(x)=$ $0, \forall x \in[a, b]$.
\end{problema}

\begin{proof}
	Por reducción al absurdo, supóngase que $f(c)\neq 0\ni \exists c\in [a,b] $. Por hipótesis, sabemos que $f\geq 0$, entonces $f(c)>0$. Ahora bien, sabemos que $f$ es una función continua. $\implies$ Por la definición, $f$ es continua en $c$, dado un $f(c)>\varepsilon >0$, existe un $\delta >0$ tal que si $x$ es un punto en $[a,b]$ que satisface $|x-c|<\delta \implies |f(x)-f(c)|<\epsilon$. $\implies$ Aplicamos el teorema aditivo\footnote{Teorema 7.2.13 de Introduction to Real Analysis de Bartle \& Sherbert - 4 edición.} tal que 
	
	$$\int_a^b f= \int_a^{c-\delta } f +\int_{c-\delta}^{c+\delta}+\int_{c+\delta}^{b}>0 \quad \text{}$$
	
Ahora bien, sabemos $f>0$, ya que $f(c)-\varepsilon>0$, pero esto quiere de $f(x)dx>0 $ y $f(x)dx=0 (\to\gets)$. $\therefore \ f(x)=0, \forall x\in [a,b]$. 

\end{proof}


%------------



\begin{problema}
	Sean $f, g$ y $h$ funciones acotadas en $[a, b]$.
	\begin{enumerate}
		\item Demuestre que si $h(x)=0$ en $[a, b]$, excepto en un número finito de puntos de $[a, b]$, entonces $h$ es Riemann integrable en $[a, b]$ y se tiene que $\int_{a}^{b} h=0$.
		\begin{proof}
			Por hipótesis, tenemos un número finito de puntos en donde $h(x)\neq 0$ en $[a,b]$. Supóngase que este número finito de puntos se puede expresar como una partición $P_n=\{s_0,s_1,s_2,s_3,\cdots, s_k\}$ tal que $a=s_0<s_1<\cdots <s_n=b$. Nótese que $|h(x)|\leq M$. Para $P_n$, proponemos encontrar su integral inferior y superior tal que 
			
			\begin{align*}
				\sup\{L(P,f)\}&=\sup\{\sum_{i=1}^{n} m_i(h)\Delta x_k\}\\
				\inf\{M(P,f)\}&=\inf\{\sum_{i=1}^{n} M_i(g)\Delta x_k\}
			\end{align*}
			
			 $\implies h $ es Riemann integrable, entonces por teorema, sabemos que existe una sucesión de particiones $\left(P_{n}\right), P_{n} \in$ $P[a, b], \forall n \in \mathbb{Z}^{+} \ni \lim _{n \rightarrow \infty}\left[U\left(f, P_{n}\right)-L\left(f, P_{n}\right)\right]=0 .$ En este caso,
			 $$
			 \int_{a}^{b} f=\lim _{n \rightarrow \infty} U\left(f, P_{n}\right)=\lim _{n \rightarrow \infty} L\left(f, P_{n}\right).
			 $$
			 
			 $$\therefore \ \int_{a}^{b} h=0.$$
		\end{proof}
		\item Demuestre que si $f$ y $g$ son Riemann integrables en $[a, b]$ y $f(x)=g(x)$ en $[a, b]$, excepto en un número finito de puntos de ese intervalo, entonces se tiene que:
		$$
		\int_{a}^{b} f=\int_{a}^{b} g
		$$
		\begin{proof}
	  Dígase que tenemos una partición $P_\varepsilon=\{x_0,x_1,x_2,x_3,\cdots, x_k\}$ tal que $a=x_0<x_1<\cdots <x_n=b$. Supóngase que el número finito de puntos en donde $f(x)\neq g(x)$ en $[a,b]$, se puede expresar como una partición $P_\varepsilon^*=\{s_0,s_1,s_2,s_3,\cdots, s_k\}$ tal que $a=s_0<s_1<\cdots <s_k=b$ y otra partición $P_\varepsilon^{**}$ que agrupa a los puntos $f(x)=g(x)$. Entonces, $P_\varepsilon=P_\varepsilon^*\cup P_\varepsilon^{**}$. Nótese que $|f(x)|\leq M$ y $|g(x)|\leq M=\fbox{ $\sup\{g(x)-g(y):x,y\in [a,b]\}$ }$. $\implies$ Como sabemos que $f$ y $g$ son Riemann integrables, por el Criterio de Cauchy, $\forall \varepsilon >0 \exists P_\varepsilon \in P[a,b] \ni P_\varepsilon \subset P$, se tiene que: 
	$$U(P_\varepsilon,f)-L(P_\varepsilon,f)<\varepsilon/2.$$
	
	Supóngase además que los intervalos de $P_\varepsilon$ están definidos como $\Delta x_k< \varepsilon/ (2nM)$. Entonces, 
	
	\begin{align*}
		\begin{aligned}
			U\left(g, P_{\epsilon}\right)-L\left(g, P_{\epsilon}\right) &=U\left(g, P_{\epsilon}^{*}\right)-L\left(g, P_{\epsilon}^{*}\right)+U\left(g, P_{\epsilon}^{**}\right)-L\left(g, P_{\epsilon}^{**}\right) \\
			&=U\left(g, P_{\epsilon}^{*}\right)-L\left(g, P_{\epsilon}^{*}\right)+U\left(f, P_{\epsilon}^{**}\right)-L\left(f, P_{\epsilon}^{**}\right) \\
			&<\frac{\epsilon}{2 nM} n M+\frac{\epsilon}{2} \\
			&=\epsilon
		\end{aligned}
	\end{align*}

	$$
	\therefore	\int_{a}^{b} f=\int_{a}^{b} g
	$$
	
		\end{proof}
	\end{enumerate}
\end{problema}







%------------


\begin{problema}
	Sea $f$ una función acotada en $[a, b]$. Suponga que $f$ es integrable en todo intervalo de la forma $[c, d]$, con $a<c<d<b$. Demuestre que $f$ es integrable en $[a, b]$.
\end{problema}
\begin{proof}
	Por hipótesis, conocemos que $f$ está acotada en $[a,b]$. Además, $f$ es integrable en todo intervalo de la forma $[c,d]$, en donde $a<c<d<b$. Proponemos una partición $P=\{x_0,x_1,\cdots, x_n\} \ni a=x_0<x_1<x_2<x_3<\cdots <x_n=b$. $\implies$ Todo intervalo de la forma $[c,d]$, que es integrable, se puede expresar como $\{[x_{i-1},x_i]\}_{i=1}^n$. $\implies$ Previamente, definimos $a=x_0$ y $b=x_n$. Por el teorema aditivo\footnote{Teorema 7.2.13 de Introduction to Real Analysis de Bartle \& Sherbert - 4 edición.}, conocemos: 
	$$\int_a^b=\int_a^c+\int_{c .}^b$$
	$\implies$ Si aplicamos el teorema aditivo a todos los intervalos sucesivamente, tenemos
	$$ \int_a^b=\int_{x_0}^{x_n}= \int_{x_0}^{x_1}+ \int_{x_1}^{x_n} f=\int_{x_0}^{x_1}f+\int_{x_1}^{x_2}f+\int_{x_2}^{x_n}f=\int_{x_0}^{x_1}f+\int_{x_1}^{x_2}f+\cdots + \int_{x_{n-1}}^{x_n}f.$$
	\begin{align}
		\implies \int_{x_0}^{x_n} f = \sum_{i=1}^{n} \int_{x_{i-1}}^{x_i}f.
	\end{align}
	$\implies$ Para demostrar que (1) es integrable en $f$ procederemos por inducción. 
	
	\linita 
	
	\fbox{Paso base}  $n=1$: 
	
	$$\int_{x_0}^{x_1} f = \int_{x_{0}}^{x_1}f \qquad (\text{Integrable por hipótesis.})$$
	
	\linita 
	
	\fbox{Paso inductivo}	Asumimos que (1) es verdadera para $n=k\geq 1$. Es decir,
	
	\begin{align*}
		\int_{x_0}^{x_k} f = \sum_{i=1}^{k} \int_{x_{i-1}}^{x_i}f
	\end{align*}
	
	 Entonces, ahora es necesario demostrar que si $n=k+1$, entonces
	\begin{align*}
		\int_{x_0}^{x_{k+1}} f = \sum_{i=1}^{k+1} \int_{x_{i-1}}^{x_i}f.
	\end{align*}
Por lo cual, 
		\begin{align*}
		\int_{x_0}^{x_{k+1}} f = \underbrace{\int_{x_0}^{x_{k}} f +\int_{x_k}^{x_{k+1}} f}_{\text{teorema aditivo}} =  \sum_{i=1}^{k} \int_{x_{i-1}}^{x_i}f+\int_{x_k}^{x_{k+1}}f=  \sum_{i=1}^{k+1} \int_{x_{i-1}}^{x_i}f. 
	\end{align*}
	$\therefore \ f$ es integrable en $[a,b]$. 
\end{proof}









%------------
\begin{problema}
	 Considere $f$ la función definida por:
	$$
	f(x)=\left\{\begin{array}{cl}
		1-x &,  x \in \mathbb{Q} \\
		x  &,  x \in \mathbb{I}rr
	\end{array}\right.
	$$
	¿Es $f$ integrable en $[0,1]$ ?
\end{problema}
\begin{sol}
	Para determinar que $f$ es integrable en $[0,1]$, debemos comprobar que: 
	$$\underline{\int_0^1}f=\overline{\int_0^1}f=\int_0^1 f.$$
	Entonces, 
	\begin{align}
		U(P,f) &= \sum_{k=1}^{k} M_k(f)\Delta x_k = \sum_{k=1}^{k} (x)\Delta x_k= x\sum_{k=1}^{k}\Delta x_k= x(1-0)= x.\\
	\begin{split}	L(P,f) & = 
			\sum_{k=1}^{k} m_k(f)\Delta x_k = \sum_{k=1}^{k} (1-x)\Delta x_k=\sum_{k=1}^{k} \Delta x_k- x\sum_{k=1}^{k}\Delta x_k=\\&= (1-0)-x(1-0)= 1-x.
		\end{split}
	\end{align}
Por (1) sabemos que, 
$$\overline{\int_0^1}f= \inf_{P\in P[0,1]}\{U(P,f)\}=0.$$

Por (2) sabemos que, 
$$\underline{\int_0^1}f= \sup_{P\in P[0,1]}\{L(P,f)\}=1.$$

$ \implies \underline{\int_0^1}f\neq \overline{\int_0^1}f. \ \therefore f$ no es integrable en $[0,1]$. 
\end{sol}
\begin{cajita}
	Un caso particular, como se demostró en clase, si $x=0$ (función de Dirichlet) entonces no es integrable.
\end{cajita}
%------------
\begin{problema}
	 Suponga que $f$ es una función acotada de valores reales sobre $[a, b]$, y que $f^{2} \in R[a, b]$. ¿Implica lo anterior que $f \in R[a, b]$ ?
\end{problema}
\begin{sol}
	Procederemos con un contraejemplo. Sean $f^2$ y $f$ definidos sobre $[a,b]$, tal que: 
	$$f(x)=\begin{cases}
		x, & x\in \mathbb{Q}\\
		-x, & x\in \mathbb{I}rr\\
	\end{cases}\implies f^2(x)= \begin{cases}
	x^2, & x\in \mathbb{Q}\\
	x^2, & x\in \mathbb{I}rr.\\
\end{cases}$$

\linita 

Comprobamos que $f^2(x)\in R[a,b]$, 
	\begin{align}
	U(P,f) &= \sum_{k=1}^{k} M_k(f)\Delta x_k = \sum_{k=1}^{k} (x^2)\Delta x_k= x^2\sum_{k=1}^{k}\Delta x_k= x^2(1-0)= x^2.\\
	L(P,f) &= \sum_{k=1}^{k} m_k(f)\Delta x_k = \sum_{k=1}^{k} (x^2)\Delta x_k= x^2\sum_{k=1}^{k}\Delta x_k= x^2(1-0)= x^2.
\end{align}
Por (1) sabemos que, 
$$\overline{\int_0^1}f= \inf_{P\in P[a,b]}\{U(P,f)\}=x^2.$$

Por (2) sabemos que, 
$$\underline{\int_0^1}f= \sup_{P\in P[a,b]}\{L(P,f)\}=x^2.$$

$$\implies \underline{\int_0^1}f=\overline{\int_0^1}f. \ \therefore f^2(x)\in R[a,b]$$

\linita 

Ahora bien, comprobamos que $f(x)\in R[a,b]$, 


	\begin{align}
	U(P,f) &= \sum_{k=1}^{k} M_k(f)\Delta x_k = \sum_{k=1}^{k} (x^2)\Delta x_k= x\sum_{k=1}^{k}\Delta x_k= x(1-0)= x.\\
	L(P,f) &= \sum_{k=1}^{k} m_k(f)\Delta x_k = \sum_{k=1}^{k} (-x)\Delta x_k= -x\sum_{k=1}^{k}\Delta x_k= -x(1-0)= -x.
\end{align}
Por (1) sabemos que, 
$$\overline{\int_0^1}f= \inf_{P\in P[a,b]}\{U(P,f)\}=x.$$

Por (2) sabemos que, 
$$\underline{\int_0^1}f= \sup_{P\in P[a,b]}\{L(P,f)\}=-x.$$

$$\implies \underline{\int_0^1}\neq \overline{\int_0^1}f.  \ \therefore  f\not\in R[0,1].$$

\linita 

Por lo que podemos concluir que si $f$ es una función acotada de valores reales sobre $[a, b]$, y que $f^{2} \in R[a, b]$, no necesariamente implica que $f \in R[a, b]$. 

\end{sol}

\begin{cajita}
	Un ejemplo sería $x=1$. 
\end{cajita}


%---------------------------
%\bibliographystyle{apa}
%\bibliography{referencias.bib}

\end{document}