\documentclass[a4paper,12pt]{article}
\usepackage[top = 2.5cm, bottom = 2.5cm, left = 2.5cm, right = 2.5cm]{geometry}
\usepackage[T1]{fontenc}
\usepackage[utf8]{inputenc}
\usepackage{multirow} 
\usepackage{booktabs} 
\usepackage{graphicx}
\usepackage[spanish]{babel}
\usepackage{setspace}
\setlength{\parindent}{0in}
\usepackage{float}
\usepackage{fancyhdr}
\usepackage{amsmath}
\usepackage{amssymb}
\usepackage{amsthm}
\usepackage[numbers]{natbib}
\newcommand\Mycite[1]{%
	\citeauthor{#1}~[\citeyear{#1}]}
\usepackage{graphicx}
\usepackage{subcaption}
\usepackage{booktabs}
\usepackage{etoolbox}
\usepackage{minibox}
\usepackage{hyperref}
\usepackage{xcolor}
\usepackage[skins]{tcolorbox}
%---------------------------

\newtcolorbox{cajita}[1][]{
	 #1
}

\newenvironment{sol}
{\renewcommand\qedsymbol{$\square$}\begin{proof}[\textbf{Solución.}]}
	{\end{proof}}

\newenvironment{dem}
{\renewcommand\qedsymbol{$\blacksquare$}\begin{proof}[\textbf{Demostración.}]}
	{\end{proof}}

\newtheorem{problema}{Problema}
\newtheorem{definicion}{Definición}
\newtheorem{ejemplo}{Ejemplo}
\newtheorem{teorema}{Teorema}
\newtheorem{corolario}{Corolario}[teorema]
\newtheorem{lema}[teorema]{Lema}
\newtheorem{prop}{Proposición}
\newtheorem*{nota}{\textbf{NOTA}}
\renewcommand\qedsymbol{$\blacksquare$}
\usepackage{svg}
\usepackage{tikz}
\usepackage[framemethod=default]{mdframed}
\global\mdfdefinestyle{exampledefault}{%
linecolor=lightgray,linewidth=1pt,%
leftmargin=1cm,rightmargin=1cm,
}




\newenvironment{noter}[1]{%
\mdfsetup{%
frametitle={\tikz\node[fill=white,rectangle,inner sep=0pt,outer sep=0pt]{#1};},
frametitleaboveskip=-0.5\ht\strutbox,
frametitlealignment=\raggedright
}%
\begin{mdframed}[style=exampledefault]
}{\end{mdframed}}
\newcommand{\linea}{\noindent\rule{\textwidth}{3pt}}
\newcommand{\linita}{\noindent\rule{\textwidth}{1pt}}

\AtBeginEnvironment{align}{\setcounter{equation}{0}}
\pagestyle{fancy}

\fancyhf{}









%----------------------------------------------------------
\lhead{\footnotesize Análisis de Variable Real 1}
\rhead{\footnotesize  Rudik Roberto Rompich}
\cfoot{\footnotesize \thepage}


%--------------------------

\begin{document}
 \thispagestyle{empty} 
    \begin{tabular}{p{15.5cm}}
    \begin{tabbing}
    \textbf{Universidad del Valle de Guatemala} \\
    Departamento de Matemática\\
    Licenciatura en Matemática Aplicada\\\\
   \textbf{Estudiante:} Rudik Roberto Rompich\\
   \textbf{Correo:}  \href{mailto:rom19857@uvg.edu.gt}{rom19857@uvg.edu.gt}\\
   \textbf{Carné:} 19857
    \end{tabbing}
    \begin{center}
        MM2034 - Análisis de Variable Real 2 - Catedrático: Dorval Carías\\
        \today
    \end{center}\\
    \hline
    \\
    \end{tabular} 
    \vspace*{0.3cm} 
    \begin{center} 
    {\Large \bf  HT 2
} 
        \vspace{2mm}
    \end{center}
    \vspace{0.4cm}
%--------------------------

\begin{problema}
	Suponga que $f \geq 0, f$ es continua en $[a, b], \mathrm{y} \int_{a}^{b} f(x) d x=0 .$ Demuestre que $f(x)=$ $0, \forall x \in[a, b]$.
\end{problema}

\begin{proof}
	Por hipótesis, sabemos que: $f\geq 0$, $f$ es continua en $[a,b]\implies $ por teorema $f\in R[a,b]$ y $\int_a^b f=0$. 
\end{proof}


%------------



\begin{problema}
	Sean $f, g$ y $h$ funciones acotadas en $[a, b]$.
	\begin{enumerate}
		\item Demuestre que si $h(x)=0$ en $[a, b]$, excepto en un número finito de puntos de $[a, b]$, entonces $h$ es Riemann integrable en $[a, b]$ y se tiene que $\int_{a}^{b} h=0$.
		\begin{proof}
			Por hipótesis, tenemos un número finito de puntos en donde $h(x)\neq 0$ en $[a,b]$. Supóngase que este número finito de puntos se puede expresar como $\{s_1,s_2,s_3,\cdots, s_k\}$.  Nótese que $|h(x)|\leq M$. 
		\end{proof}
		\item Demuestre que si $f$ y $g$ son Riemann integrables en $[a, b]$ y $f(x)=g(x)$ en $[a, b]$, excepto en un número finito de puntos de ese intervalo, entonces se tiene que:
		$$
		\int_{a}^{b} f=\int_{a}^{b} g
		$$
		\begin{proof}
		Por hipótesis, tenemos un número finito de puntos en donde $f(x)\neq g(x)$ en $[a,b]$. Supóngase que este número finito de puntos se puede expresar como $\{s_1,s_2,s_3,\cdots, s_k\}$. Nótese que $|f(x)|\leq M$ y $|g(x)|\leq M$. 
		\end{proof}
	\end{enumerate}
\end{problema}







%------------


\begin{problema}
	Sea $f$ una función acotada en $[a, b]$. Suponga que $f$ es integrable en todo intervalo de la forma $[c, d]$, con $a<c<d<b$. Demuestre que $f$ es integrable en $[a, b]$.
\end{problema}
\begin{proof}
	C.
\end{proof}









%------------
\begin{problema}
	 Considere $f$ la función definida por:
	$$
	f(x)=\left\{\begin{array}{cl}
		1-x &,  x \in \mathbb{Q} \\
		x  &,  x \in \mathbb{I}rr
	\end{array}\right.
	$$
	¿Es $f$ integrable en $[0,1]$ ?
\end{problema}
\begin{sol}
	Para determinar que $f$ es integrable en $[0,1]$, debemos comprobar que: 
	$$\underline{\int_0^1}f=\overline{\int_0^1}f=\int_0^1 f.$$
	Entonces, 
	\begin{align}
		U(P,f) &= \sum_{k=1}^{k} M_k(f)\Delta x_k = \sum_{k=1}^{k} (x)\Delta x_k= x\sum_{k=1}^{k}\Delta x_k= x(1-0)= x.\\
	\begin{split}	L(P,f) & = 
			\sum_{k=1}^{k} m_k(f)\Delta x_k = \sum_{k=1}^{k} (1-x)\Delta x_k=\sum_{k=1}^{k} \Delta x_k- x\sum_{k=1}^{k}\Delta x_k=\\&= (1-0)-x(1-0)= 1-x.
		\end{split}
	\end{align}
Por (1) sabemos que, 
$$\overline{\int_0^1}f= \inf_{P\in P[a,b]}\{U(P,f)\}=1-x.$$

Por (2) sabemos que, 
$$\underline{\int_0^1}f= \sup_{P\in P[a,b]}\{L(P,f)\}=x.$$

$ \implies \underline{\int_0^1}f\neq \overline{\int_0^1}f. \ \therefore f$ no es integrable en $[0,1]$. 
\end{sol}
\begin{cajita}
	Un caso particular, como se demostró en clase, si $x=0$ (función de Dirichlet) entonces no es integrable.
\end{cajita}
%------------
\begin{problema}
	 Suponga que $f$ es una función acotada de valores reales sobre $[a, b]$, y que $f^{2} \in R[a, b]$. ¿Implica lo anterior que $f \in R[a, b]$ ?
\end{problema}
\begin{sol}
	Procederemos con un contraejemplo. Sean $f^2$ y $f$ definidos sobre $[a,b]$, tal que: 
	$$f(x)=\begin{cases}
		x, & x\in \mathbb{Q}\\
		-x, & x\in \mathbb{I}rr\\
	\end{cases}\implies f^2(x)= \begin{cases}
	x^2, & x\in \mathbb{Q}\\
	x^2, & x\in \mathbb{I}rr.\\
\end{cases}$$

\linita 

Comprobamos que $f^2(x)\in R[a,b]$, 
	\begin{align}
	U(P,f) &= \sum_{k=1}^{k} M_k(f)\Delta x_k = \sum_{k=1}^{k} (x^2)\Delta x_k= x^2\sum_{k=1}^{k}\Delta x_k= x^2(1-0)= x^2.\\
	L(P,f) &= \sum_{k=1}^{k} m_k(f)\Delta x_k = \sum_{k=1}^{k} (x^2)\Delta x_k= x^2\sum_{k=1}^{k}\Delta x_k= x^2(1-0)= x^2.
\end{align}
Por (1) sabemos que, 
$$\overline{\int_0^1}f= \inf_{P\in P[a,b]}\{U(P,f)\}=x^2.$$

Por (2) sabemos que, 
$$\underline{\int_0^1}f= \sup_{P\in P[a,b]}\{L(P,f)\}=x^2.$$

$$\implies \underline{\int_0^1}f=\overline{\int_0^1}f. \ \therefore f^2(x)\in R[a,b]$$

\linita 

Ahora bien, comprobamos que $f(x)\in R[a,b]$, 


	\begin{align}
	U(P,f) &= \sum_{k=1}^{k} M_k(f)\Delta x_k = \sum_{k=1}^{k} (x^2)\Delta x_k= x\sum_{k=1}^{k}\Delta x_k= x(1-0)= x.\\
	L(P,f) &= \sum_{k=1}^{k} m_k(f)\Delta x_k = \sum_{k=1}^{k} (-x)\Delta x_k= -x\sum_{k=1}^{k}\Delta x_k= -x(1-0)= -x.
\end{align}
Por (1) sabemos que, 
$$\overline{\int_0^1}f= \inf_{P\in P[a,b]}\{U(P,f)\}=x.$$

Por (2) sabemos que, 
$$\underline{\int_0^1}f= \sup_{P\in P[a,b]}\{L(P,f)\}=-x.$$

$$\implies \underline{\int_0^1}\neq \overline{\int_0^1}f.  \ \therefore  f\not\in R[0,1].$$

\linita 

Por lo que podemos concluir que si $f$ es una función acotada de valores reales sobre $[a, b]$, y que $f^{2} \in R[a, b]$, no necesariamente implica que $f \in R[a, b]$. 

\end{sol}



%---------------------------
%\bibliographystyle{apa}
%\bibliography{referencias.bib}

\end{document}